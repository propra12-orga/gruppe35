\documentclass{beamer}

\usepackage[ngerman]{babel}
\usepackage[utf8]{inputenc} 
\usepackage{graphicx}
\graphicspath{{logos/}{figures/}}
\usepackage{framed} %Rahmen Umgebung

\usepackage{amsmath}
\usepackage{verbatim}

\usetheme{CambridgeUS}

\usepackage{listings} %Programmcode schön einfügen
\lstloadlanguages{XML}
\lstset{language=XML,
extendedchars=true,
basicstyle=\tiny,
tabsize=2,
xleftmargin=20pt,
breakatwhitespace=true,
columns=fullflexible,
keepspaces=true}
%\renewcommand{\lstlistingname}{Code}


\setbeamercovered{transparent}
\beamertemplatenavigationsymbolsempty
\setbeamertemplate{footline}[frame number]

%Show Outline at begin of sections
\AtBeginSection[]{%
  \begin{frame}[plain]
    \frametitle{Gliederung}
    \tableofcontents[sectionstyle=show/shaded,subsectionstyle=shaded/show/shaded]
  \end{frame}
  \addtocounter{framenumber}{-1}% If you don't want them to affect the slide number
}

%Title, subtitle, authors, date, affiliation and logos
\title{Austauschbare Datenformate \& Parsing}
\subtitle{Referat im Rahmen des Programmierpraktikums 2012} 
\author{Peet Cremer}
\institute{%
Lehrstuhl für Physik der weichen Materie \\
Heinrich-Heine-Universit\"at D\"usseldorf, \\
Universit\"atsstra\ss e 1, D-40225 D\"usseldorf, Germany
}%
\date{6. Juli 2012}

%References
\bibliographystyle{abbrvunsrt2}

\begin{document}

%Titlepage
\begin{frame}[plain]
  %Logos
  \begin{columns}
    \column{0.6\textwidth}
    \column{0.4\textwidth}
      \begin{figure}
        \centering
        \includegraphics[width=0.65\columnwidth]{hhu_logo.pdf}
      \end{figure}
  \end{columns}
  
  \titlepage
  
  \addtocounter{framenumber}{-1}
\end{frame}

%Inhaltsverzeichnis
\begin{frame}[plain]
  \frametitle{Gliederung}
  \tableofcontents[sectionstyle=show/show,subsectionstyle=show/show/show]
  \addtocounter{framenumber}{-1}
\end{frame}
\addtocounter{framenumber}{-1}

\section{Einleitung}
\addtocounter{framenumber}{+1}

\begin{frame}{Datenformate}
  \begin{itemize}
    \item  Plattformunabhängige Datenformate besonders für das Internet wichtig
    \item  Hierarchische Baumstrukturen 
    \item  Menschenlesbar
    \item  Müssen geschrieben und ausgelesen werden: Parser
  \end{itemize}
\end{frame}

\section{XML}

\begin{frame}{XML}
  \begin{itemize}
    \item  e\textbf{X}tensible \textbf{M}arkup \textbf{L}anguage
    \item  Herausgegeben vom World Wide Web Consortium (W3C)
    \item  Darstellung hierarchisch strukturierter Daten in Textdateien
    \item  ASCII codiert und damit menschenlesbar
    \item  Definiert strukturelle und inhaltliche Einschränkungen um als Grundlage für 	anwendungsspezifische Sprachen zu dienen
    \item  Beispiele für XML--Sprachen: RSS, MathML, GraphML, XHTML, XAML, SVG
  \end{itemize}
\end{frame}

\begin{frame}{Inhalte}
  \begin{itemize}
    \item  Elemente beginnen mit Start--Tag enden mit End--Tag
    \item  Attribute, die zu einem Element gehören
    \item  Verarbeitungsanweisungen
    \item  Kommentare
  \end{itemize}
\end{frame}


\begin{frame}[fragile]{XML--Syntax}
\begin{lstlisting}
<?xml version="1.0" encoding="UTF-8" standalone="no"?>
   <Level1 xsize="3" ysize="3">
      <fields>
         <field type="f" x="1" y="1"/>
         <field type="f" x="1" y="2"/>
         <field type="e" x="1" y="3"/>
         <field type="f" x="2" y="1"/>
         <field type="s" x="2" y="2"/>
         <field type="e" x="2" y="3"/>
         <field type="e" x="3" y="1"/>
         <field type="e" x="3" y="2"/>
         <field type="e" x="3" y="3"/>
      </fields>
      <spawnpoints>
         <spawnpoint x="1" y="1"/>
      </spawnpoints>
      <exits>
         <exit x="3" y="3"/>
      </exits>
      <Gelaber>
         BlaBla
      <\Gelaber>
      <?Ziel--Name Parameter ?>
      <!-- Kommentar -->
   </Level1>
\end{lstlisting}
\end{frame}

\begin{frame}{XML--Regeln}
  XML--Dokument heißt \alert{wohlgeformt} wenn es die XML--Regeln einhält
  \begin{itemize}
    \item  Es gibt genau ein \alert{Wurzelelement} (das äußerste Element der Hierarchie)
    \item  Alle Elemente besitzen ein \alert{Begin--Tag} und ein \alert{End--Tag} und den Inhalt dazwischen	   
    \item  Elemente am ohne Unterelemente können alternativ auch in sich geschlossen werden 
    \item  Elemente enthalten nur ein Attribut mit demselben Namen
  \end{itemize}
\end{frame}

\section{Parser}

\begin{frame}{Parser}
XML--Dokumente sollen automatisch generiert und ausgelesen werden
\begin{itemize}
  \item  \textbf{D}ocument \textbf{O}bject \textbf{M}odel
    \begin{itemize}
      \item  Das Ganze Dokument wird in den Speicher geladen
      \item  Zugriff auf das Dokument als Baumstruktur
      \item  Erlaubt Manipulation und Zurückschreiben in das XML--Dokument
      \item  Speicherintensiv
      \item  Für kleine Dokumente gut geeignet
    \end{itemize}
  \item  \textbf{S}imple \textbf{A}PI for \textbf{X}ML
    \begin{itemize}
      \item  XML--Dokument als sequentieller Datenstrom
      \item  Auf Ereignisse im Datenstrom wird reagiert
      \item  Wenig Speicherintensiv
      \item  Keine Manipulation möglich
      \item  Für große Dokumente gut geeignet
    \end{itemize}
\end{itemize}
\end{frame}






%%%COMMENT%%%
\begin{comment}
%Appendix
\appendix
\newcounter{finalframe}
\setcounter{finalframe}{\value{framenumber}}

%Quellen
\begin{frame}[plain]{References}
	\bibliography{references.bib}
\end{frame}



\setcounter{framenumber}{\value{finalframe}}
\end{comment}
%%%COMMENT%%%
\end{document}